% IPH-UFRGS Beamer Theme v.01
% by Márcio Shigueaki Inada <mshigue@gmail.com>

\documentclass[
    brazil,
    %aspectratio=169, % altera para o formato da apresentação para 16:9
    ]{beamer}

\usetheme{UFRGS} % use beamerthemeUFRGS.sty

% {{ ---- Pacotes ----
\usepackage[portuguese]{babel}
\usepackage[utf8]{inputenc}
\usepackage[T1]{fontenc}

\usepackage[
    alf,                % citação alfabética autor data
    abnt-etal-list=0,   % lista nomes na referência
    abnt-etal-cite=3,   % número de autores et al
    abnt-etal-text=it   % et al em itálico
]{abntex2cite}	        % Citações padrão ABNT
% ---- }}

% {{ ---- Configurações ----
%\definecolor{beamer@ufrgsThemeText}{RGB}{3,95,140} % altera a cor do tema UFRGS
%\setbeamerfont{frametitle}{size=\Large} % altera o tamanho da fonte do título
% ---- }}

% {{ ---- Informações ----
\title[titulo curto]{Titulo do trabalho}
\subtitle[subtitulo curto]{Subtitulo do trabalho}
\author[nome curto]{Seu nome completo}
\institute[UFRGS]{orientador: nome do orientador \\ Universidade Federal do Rio Grande do Sul}
\date{Junho 2019}
% ---- }}

% {{ ---- Documento ----
\begin{document}

\begin{frame}[plain,noframenumbering]
  \titlepage
\end{frame}

\begin{frame}{Sumário}
    \tableofcontents
\end{frame}

\section{Introdução}
\begin{frame}{Introdução}
    Texto normal \par
    \textbf{Texto negrito} \par
    \textit{Texto itálico} \par
    \alert{Texto alerta} \par
\end{frame}

\section{Listas}
\subsection{Itens}
\begin{frame}{Itens}
    \begin{itemize}
        \item Como?
        \begin{itemize}
            \item como pensar?
            \begin{itemize}
                \item como agir?
            \end{itemize}        
        \end{itemize}
        \item Onde?
        \item Quando?
    \end{itemize}
\end{frame}

\subsection{Numerados}
\begin{frame}[t]{Numerados}
    \textbf{\Large \textcolor{beamer@ufrgsThemeText}{Titulo de um texto no topo}}
    \vspace{1cm}
    \begin{enumerate}
        \item Como?
        \item Onde?
        \item Quando?
    \end{enumerate}
\end{frame}

\section{Blocos}
\begin{frame}{Blocos}
    \begin{block}{}
        Caixa sem título
    \end{block}
    \begin{block}{Caixa}
        Caixa normal
    \end{block}
    \begin{alertblock}{Caixa de alerta}
        Caixa de alerta
    \end{alertblock}
\end{frame}

\section{Citações}
\begin{frame}{Citações}
    The Tragedy of the Commons \cite{Hardin1968}. \par
    \citeonline{Hardin1968} em \textit{The Tragedy of the Commons}. \par
    Endogenous technological change \cite{romer1990endogenous}.
\end{frame}

\section{Referências}
\begin{frame}{Referências}
    \footnotesize
    \bibliography{reference.bib} % Arquivo BibTex
\end{frame}

\end{document}
% ---- }}
